%%%%%%%%%%%%%%%%%%%%%%%%%%%%%%%%%%%%%%%%%
% Twenty Seconds Resume/CV
% LaTeX Template
% Version 1.1 (8/1/17)
%
% This template has been downloaded from:
% http://www.LaTeXTemplates.com
%
% Original author:
% Carmine Spagnuolo (cspagnuolo@unisa.it) with major modifications by 
% Vel (vel@LaTeXTemplates.com)
%
% some more modification by : Ayush Gupta 
% ag5613
%%%%%%%%%%%%%%%%%%%%%%%%%%%%%%%%%%%%%%%%%

%----------------------------------------------------------------------------------------
%	PACKAGES AND OTHER DOCUMENT CONFIGURATIONS
%----------------------------------------------------------------------------------------

\documentclass[letterpaper]{twentysecondcv} % a4paper for A4

%----------------------------------------------------------------------------------------
%	 PERSONAL INFORMATION
%----------------------------------------------------------------------------------------

% If you don't need one or more of the below, just remove the content leaving the command, e.g. \cvnumberphone{}
\profilepic{}
\cvname{Ayush Gupta} % Your name
\cvjobtitle{Student} % Job title/career
\cvsite{github.com/ag5613} %any website
\cvdate{19 December, 2000} % Date of birth
\cvaddress{Agra, Uttar Pradesh, INDIA} % Short address/location, use \newline if more than 1 line is required
\cvnumberphone{+91 9012887831} % Phone number
%\cvsite{https://www.facebook.com/ridoy89} % Facebook Profile
\cvmail{ayushgupta5613@gmail.com  ayush.dei@protonmail.com}% Email address
\begin{document}
%----------------------------------------------------------------------------------------
%	 ABOUT ME
%----------------------------------------------------------------------------------------
\aboutme{}
%\To have no About Me section, just remove all the text and leave \aboutme{}

% this is used for the side bar. This (Aside) was added into the template 
\begin{aside}
\vspace{6cm}
\profilesection{Programming}
\newline
\includegraphics[height=4.25cm]{pie.PNG}
\profilesection{Modules}
OpenCV
Tesseract
tkinter
PyTorch
Matplotlib
scikit-Learn
NumPy
\profilesection{Softwares}
Arduino IDE
MySQL
\LaTeX
Microsoft Excel
Microsoft Word
Microsoft Viso
Photoshop
\profilesection{Soft Skills}
Adaptable 
Hardworking
Fast Learner
Positive Attitude 
Social Interaction
Team Work 
Effective-Communication
\end{aside}
\makeprofile
%\makeprofile % Print the sidebar

%	 EDUCATION

\section{Education}
\begin{twenty} % Environment for a list with descriptions
	\twentyitem{2018-2021}{Bachelor of Vocation (B.Voc.) {\normalfont in Robotics and Artificial Intelligence }}{ CGPA: 9.05/10}{\emph{Dayalbagh Educational Institute, Agra\tab} }
	\twentyitem{2017-2018}{AISSCE, Percentage : 86\%}{}{\emph{Assisi Convent School, Etah}}
	\twentyitem{2015-2016}{AISSE, CGPA:9.6/10}{}{\emph{Assisi Convent School, Etah}}
	%\twentyitem{<dates>}{<title>}{<location>}{<description>}
\end{twenty}

% Interests 

\section{Interests}
Robotics, Automation, Machine Learning, Computer Vision, Artificial Inteligence, Arduino.

%	 Awards and Achievements

\section{Awards and Achievements}
\begin{itemize}
    \item \textbf{Secured}  \textit{Rank-6692th} \textbf{in} \textit{Google Hash Code-2020} \textbf{Online Qualification Round}
    \item \textbf{Secured}  \textit{School Rank - 1} \textbf{and} \textit{Zonal Rank 51} \textbf{in} \textit{National Computer Olympiad (NCO) -2017} \textbf{organised by SOF World}
\end{itemize}

%Course Subjects 

\section{Course Subjects}
\newline
\begin{center}
\begin{tabular}{p{5.5cm}|p{5.5cm}}
\hline
\hline
    Machine Learning, & Digital Image Processing,  \\
      Database Management System, & Operating System,\\
      Data Structures in Python,& Robotics,\\
     Programming in Python, & Data Visualisation,\\
     Microprocessors \& Microcontrollers, &  Introduction to Artificial Intelligence\\
     \hline
     \hline
\end{tabular}
\end{center}
\newline

% PROJECTS

\section{Projects}
\subsection{Minor Project (Graduation Subject)}
\begin{itemize}
    \item {RRP Configuration based manipulator}
    \item[]\textbf{Sep 2019 -- Dec 2019}
    \item[] \textbf{RRP Configuration based Robotic Manipulator for Drawing application.The manipulator was a 3-DOF planar manipulator and was able to draw basic shapes.It uses a Arduino microcontroller and three servo motors for providing joint actuation.}
    \item {Gesture Controlled Robotic car with Robotic Arm}
    \item[] \textbf{Feb 2019 -- May 2019 }
    \item[] \textbf{Hand Gesture controlled robotic car with embedded robotic arm. It comprises of two modules, one is the car and other is the robotic arm. The movements of the car and that of Arm was controlled by the gestures of the hand. The gestures of the hands was recognized via flex sensors and sent to the receiver module. The main components of the project was Arduino, Flex Sensor, IR transmitter and Receiver module, Robotic Arm, Robotic Car}
    \item {Smart Street-light System}
    \item[] \textbf{Sep 2018 -- Dec 2018}
    \item[] \textbf{A Project to demonstrate the energy saving model by turning ON the lights only when it is needed. The working of the system was that the highway streetlight will be turned ON only when the vehicle comes. It uses different sensors to detect the vehicle and the microcontroller sends the signal to Turn ON subsequent lights.}
\end{itemize}
\newpage % For new Page
\makeprofile % for the sidebar
\subsection{ Personal Projects}
\begin{itemize}
\item {Face-Recognition based Attendance System}
    \item[] \textbf{Oct 2019 -- Nov 2019}
    \item[] \textbf{Project programmed in Python Programming language which Detects and recognize  people from a live video and mark them present in that particular lecture. The dataset used in this project was the photos of the classmates and were trained in the model for face recognition.}
    \item[] Accuracy : 92\%
\item {Arduino Based Environment Thermometer}
\item[] \textbf{Sep 2019}
\item[] \textbf{Arduino Based Environment Thermometer which senses the Temperature and Humidity of the surrounding environment. It uses a Arduino Microcontroller, DHT-11 Temperature and Humidity Sensor to sense the temperature and humidity and a 16X2 LCD display to show the output to the user.}
\end{itemize}

% Training and Certification 

\section{Training/Certification}
\begin{itemize}
\item{
\textbf{Attended} \textit{"Robotics Workshop"} \textbf{organized by} \textit{Robolab Technologies} \textbf{in }\textit{Dayalbagh Educational Institute, Agra } \textbf{and acquired adequate knowledge on PLC (Programmable Logic Controller), Actuators, Control System,Drive System, Pistons (Pneumatic Actuator), Quadcopter, Robotic Arm and other Robots.}}
\item{\textbf{NPTEL Online Certification on} \emph{“Fuzzy Logics and Neural Network”}}
\item[] 
\end{itemize}

% Seminar/Paper Presentations

\section{Seminar/Paper Presentations}
\begin{itemize}
\item{
\textbf{Presented a paper on} \emph{"A Comparison of Face Recognition Algorithms and developing a
novel system for automated Attendance"} \textbf{in Paritantra 2019 Conference}}
\item
{\textbf{Presented a seminar on} \emph{"Data Tampering: Cookies, URL and POST data"} \textbf{in B.Voc. 3rd sem.}}
\end{itemize}
%	 NEXT PAGE EXAMPLE
%\newpage % Start a new page
%\makeprofile % Print the sidebar
%\section{Other information}
%\subsection{Review}
\end{document} 
